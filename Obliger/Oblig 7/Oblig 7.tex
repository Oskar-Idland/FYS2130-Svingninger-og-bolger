\documentclass{article}
\usepackage{amsmath}
\usepackage[mathletters]{ucs}
\usepackage[utf8x]{inputenc}
\usepackage[margin=1.5in]{geometry}
\usepackage{enumerate}
\newtheorem{theorem}{Theorem}
\usepackage[dvipsnames]{xcolor}
\usepackage{pgfplots}
\setlength{\parindent}{0cm}
\usepackage{graphics}
\usepackage{graphicx} % Required for including images
\usepackage{subcaption}
\usepackage{bigintcalc}
\usepackage{pythonhighlight} %for pythonkode \begin{python}   \end{python}
\usepackage{appendix}
\usepackage{arydshln}
\usepackage{physics}
\usepackage{tikz-cd}
\usepackage{booktabs} 
\usepackage{adjustbox}
\usepackage{mdframed}
\usepackage{relsize}
\usepackage{physics}
\usepackage[thinc]{esdiff}
\usepackage{fixltx2e}
\usepackage{esint}  %for lukket-linje-integral
\usepackage{xfrac} %for sfrac
\usepackage{hyperref} %for linker, må ha med hypersetup
\usepackage[noabbrev, nameinlink]{cleveref} % to be loaded after hyperref
\usepackage{amssymb} %\mathbb{R} for reelle tall, \mathcal{B} for "matte"-font
\usepackage{listings} %for kode/lstlisting
\usepackage{verbatim}
\usepackage{graphicx,wrapfig,lipsum,caption} %for wrapping av bilder
\usepackage{mathtools} %for \abs{x}
\usepackage[norsk]{babel}
\definecolor{codegreen}{rgb}{0,0.6,0}
\definecolor{codegray}{rgb}{0.5,0.5,0.5}
\definecolor{codepurple}{rgb}{0.58,0,0.82}
\definecolor{backcolour}{rgb}{0.95,0.95,0.92}
\lstdefinestyle{mystyle}{
    backgroundcolor=\color{backcolour},   
    commentstyle=\color{codegreen},
    keywordstyle=\color{magenta},
    numberstyle=\tiny\color{codegray},
    stringstyle=\color{codepurple},
    basicstyle=\ttfamily\footnotesize,
    breakatwhitespace=false,         
    breaklines=true,                 
    captionpos=b,                    
    keepspaces=true,                 
    numbers=left,                    
    numbersep=5pt,                  
    showspaces=false,                
    showstringspaces=false,
    showtabs=false,                  
    tabsize=2
}

\lstset{style=mystyle}
\author{Oskar Idland}
\title{Oblig 7}
\date{}
\begin{document}
\maketitle
\newpage
\section*{Oppgave 2}

Vi bruker linseformelen gitt ved
\[
\frac{1}{s} + \frac{1}{s'} = \frac{1}{f}
\]
hvor $s$ er avstanden mellom vennen og linsen, $s'$ er avstanden mellom linsen og bildeplanet og $f$ er brennvidden til linsen. Vi løser for $s'$.
\[
s' = \frac{sf}{s - f} = \frac{3.5\text{ m } ⋅ 85 \text{ nm}}{3.6 \text{ m } - 85 \text{ nm}} = 87.1 \text{ nm}
\]
Videre skal bilde tas i landskapsformat med bredde $24 × 36$ mm og sjekker om hele personen får plass i bildet. Utrykket for forminskningen er definert som følger. 
\[
\left|M\right| = \frac{s'}{s} = \frac{87.1 \text{ nm}}{3.6 \text{ m}} = 2.4 ⋅ 10^{-2}
\]
Dette gir oss at personens høyde på bildet $h'$ blir 
\[
h' = \left|M\right|h = 2.4 ⋅ 10^{-2} ⋅ 1.61 \text{ m} = 38.0 \text{ mm}
\]
Vi ser da at personen ikke for plass ettersom den er høyere enn bildebrikken tillater. Vi ser at $15.8 / 38.9 = 40 \%$ av personen får plass på bildet.

\section*{Oppgave 3}
\subsection*{a)}
Ettersom et normalt øye har et nærpunkt på en avstand $s = 25$cm. Vi finner videre den optiske styrken $D$. 
\[
\frac{1}{f} = \frac{1}{s} + \frac{1}{s'} = \frac{1}{0.25 \text{ m}} + \frac{1}{0.02 \text{ m}} = 54 \text{ D}
\]
En brille med linse på $1.5$D vil justere dette til en optiske styrke på $52.5$D. Da kan vi finne det egentlige nærpunktet $s$ gitt ved 
\[
s = \frac{s'f}{s' - f} = \frac{0.02 \text{ m } ⋅ 1 /52.5 \text{ D}}{0.02 \text{ m} - 1 /52.5 \text{ D}} = 40 \text{ cm}
\]

\subsubsection*{b)}
Ettersom et normalt øye sitt fjernpunkt er uendelig lagt unna kan vi finne den optiske styrken som følger. 
\[
\frac{1}{f} = \frac{1}{s'} = \frac{1}{0.02 \text{ m}} = 50 \text{ D}
\]
En brille med linsestyrke på $-0.5$D betyr øyet har en optisk styrke på $50.5$D. Da kan vi finne det egentlige fjernpunktet $s$ gitt ved
\[
s = \frac{s'f}{s' - f} = \frac{0.02 \text{ m } ⋅ 1 /50.5 \text{ D}}{0.02 \text{ m} - 1 /50.5 \text{ D}} = 2 \text{ m}
\]

\section*{Oppgave 4}
\subsubsection*{a)}
For å få et skarpt bilde må lysstrålen fra objektet møtes ved okularets brennvidde $f_2 = 21$ mm. Avstanden mellom objektivet og okularet er $s_1' + f_2 = 19.6$ cm. Da kan vi beregne avstanden $s_1'$ fra objektivet til okularet.
\[
s_1' = 19.6 - 2.1 \text{ cm} = 17.5 \text{ cm}
\]
Til slutt bruker vi linseformelen og objektivets brennvidde $f_1 = 8$ nm for å finne avstanden $s_1$ mellom objektet og objektivet. 
\[
s_1 = \frac{s_1' f_1}{s_1'  -f_1} = \frac{17.5 \text{ cm} ⋅ 8 \text{ mm}}{17.5 \text{cm} - 8 \text{ mm}} = 8.4 \text{ mm}
\]

\subsubsection*{b)}
Den lineære forstørrelsen $M_1$ til objektive alene er gitt ved 
\[
M_1 = \frac{s_1'}{s_1} = \frac{17.5 \text{ cm}}{8.4 \text{ mm}} = 20.9 
\]

\subsubsection*{c)}
Forstørrelsen $M_2$ fra okularet alene er gitt ved
\[
M_2 = \frac{l}{f_2} = \frac{19.6 \text{ cm}}{21 \text{ mm}} = 9.3
\]
der $l$ er avstanden mellom okularet og objektivet. 

\subsubsection*{d)}
Den totale forstørrelsen $M_{\text{tot}}$ er naturligvis produktet av av forstørrelsene $M_1$ og $M_2$ fra objektivet og okularet respektivt. 
\[
M_{\text{tot}} = M_1 ⋅ M_2 = \frac{ls_1'}{f_2s_1}
\]

\subsubsection*{e)}
Den totale forstørrelsen $M_{\text{tot}}$ til dette mikroskopet blir da følgende. 
\[
M_{\text{tot}} = M_1 ⋅ M_2 = 20.9 ⋅ 9.3 ≈ 195. 
\]

\end{document}