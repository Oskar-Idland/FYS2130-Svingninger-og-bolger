\documentclass[norsk]{article}
\usepackage{amsmath}
\usepackage[mathletters]{ucs}
\usepackage[utf8x]{inputenc}
\usepackage[margin=1.5in]{geometry}
\usepackage{enumerate}
\newtheorem{theorem}{Theorem}
\usepackage[dvipsnames]{xcolor}
\usepackage{pgfplots}
\setlength{\parindent}{0cm}
\usepackage{graphics}
\usepackage{graphicx} % Required for including images
\usepackage{subcaption}
\usepackage{bigintcalc}
\usepackage{pythonhighlight} %for pythonkode \begin{python}   \end{python}
\usepackage{appendix}
\usepackage{arydshln}
\usepackage{physics}
\usepackage{tikz-cd}
\usepackage{booktabs} 
\usepackage{adjustbox}
\usepackage{mdframed}
\usepackage{relsize}
\usepackage{physics}
\usepackage[thinc]{esdiff}
\usepackage{fixltx2e}
\usepackage{esint}  %for lukket-linje-integral
\usepackage{xfrac} %for sfrac
\usepackage[colorlinks]{hyperref} %for linker, må ha med hypersetup
\usepackage[noabbrev, nameinlink]{cleveref} % to be loaded after hyperref
\usepackage{amssymb} %\mathbb{R} for reelle tall, \mathcal{B} for "matte"-font
\usepackage{listings} %for kode/lstlisting
\usepackage{verbatim}
\usepackage{graphicx,wrapfig,lipsum,caption} %for wrapping av bilder
\usepackage{mathtools} %for \abs{x}
\usepackage[norsk]{babel}
\definecolor{codegreen}{rgb}{0,0.6,0}
\definecolor{codegray}{rgb}{0.5,0.5,0.5}
\definecolor{codepurple}{rgb}{0.58,0,0.82}
\definecolor{backcolour}{rgb}{0.95,0.95,0.92}
\lstdefinestyle{mystyle}{
    backgroundcolor=\color{backcolour},   
    commentstyle=\color{codegreen},
    keywordstyle=\color{magenta},
    numberstyle=\tiny\color{codegray},
    stringstyle=\color{codepurple},
    basicstyle=\ttfamily\footnotesize,
    breakatwhitespace=false,         
    breaklines=true,                 
    captionpos=b,                    
    keepspaces=true,                 
    numbers=left,                    
    numbersep=5pt,                  
    showspaces=false,                
    showstringspaces=false,
    showtabs=false,                  
    tabsize=2
}

\lstset{style=mystyle}
\author{Oskar Idland}
\title{FYS2130 - Oblig 4}
\date{}
\begin{document}
\maketitle
\newpage
\section*{Oppgave 1}
Vi bruker definisjonen av desibel skalaen. 
\[
X = 10\log_{10}\left(\frac{I}{I_0}\right)
\] 
Hvor $I_0$ er intensiteten til en referanse, og $I$ er intensiteten til den målte strålingen. Referanse intensiteten vil da ha en desibel verdi $X_0$ på 0 som vi ser \cref{eq: X_0}. 
\begin{equation}\label{eq: X_0}
  X_0 = 10 \text{ dB} \log \left(\frac{I_0}{I_0}\right) = 0
\end{equation}
En økning i lydstyrke på $X$ desibel ved en lyd-intensitet $k I_0$på  kan vi skrive som følgende i \cref{eq: X}.
\begin{equation}\label{eq: X}
  X = 10 \text{ dB} \log \left(\frac{kI_0}{I_0}\right) = 10 \text{ dB} \log \left(k\right)
\end{equation}

\section*{Oppgave 2}
\subsection*{a)}
Vi bruker at forholdet mellom lengde og frekvens til å finne ut hvor det femte båndet ligger. 
\[
\frac{110 \text{ Hz}}{146.83 \text{ Hz}} = \frac{x \text{ cm}}{65 \text{ cm}} ⇒ x = 48.7 \text{ cm}
\]


\end{document}