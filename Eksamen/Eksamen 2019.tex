\documentclass{article}
\usepackage{amsmath}
\usepackage[mathletters]{ucs}
\usepackage[utf8x]{inputenc}
\usepackage[margin=1.5in]{geometry}
\usepackage{enumerate}
\newtheorem{theorem}{Theorem}
\usepackage[dvipsnames]{xcolor}
\usepackage{pgfplots}
\pgfplotsset{compat=1.18}
\setlength{\parindent}{0cm}
\usepackage{graphics}
\usepackage{graphicx} % Required for including images
\usepackage{subcaption}
\usepackage{bigintcalc}
\usepackage{pythonhighlight} %for pythonkode \begin{python}   \end{python}
\usepackage{appendix}
\usepackage{arydshln}
\usepackage{physics}
\usepackage{tikz-cd}
\usepackage{booktabs} 
\usepackage{adjustbox}
\usepackage{mdframed}
\usepackage{relsize}
\usepackage{physics}
\usepackage[thinc]{esdiff}
\usepackage{fixltx2e}
\usepackage{esint}  %for lukket-linje-integral
\usepackage{xfrac} %for sfrac
\usepackage{hyperref} %for linker, må ha med hypersetup
\usepackage[noabbrev, nameinlink]{cleveref} % to be loaded after hyperref
\usepackage{amssymb} %\mathbb{R} for reelle tall, \mathcal{B} for "matte"-font
\usepackage{listings} %for kode/lstlisting
\usepackage{verbatim}
\usepackage{graphicx,wrapfig,lipsum,caption} %for wrapping av bilder
\usepackage{mathtools} %for \abs{x}
\usepackage[norsk]{babel}
\definecolor{codegreen}{rgb}{0,0.6,0}
\definecolor{codegray}{rgb}{0.5,0.5,0.5}
\definecolor{codepurple}{rgb}{0.58,0,0.82}
\definecolor{backcolour}{rgb}{0.95,0.95,0.92}
\pagecolor[rgb]{0.075,0.075,0.075} \color[rgb]{1,1,1} %TODO: Slett når ferdig%
\lstdefinestyle{mystyle}{
    backgroundcolor=\color{backcolour},   
    commentstyle=\color{codegreen},
    keywordstyle=\color{magenta},
    numberstyle=\tiny\color{codegray},
    stringstyle=\color{codepurple},
    basicstyle=\ttfamily\footnotesize,
    breakatwhitespace=false,         
    breaklines=true,                 
    captionpos=b,                    
    keepspaces=true,                 
    numbers=left,                    
    numbersep=5pt,                  
    showspaces=false,                
    showstringspaces=false,
    showtabs=false,                  
    tabsize=2
}

\lstset{style=mystyle}
\author{Oskar Idland}
\title{Eksamen 2019}
\date{}
\begin{document}
\maketitle
\newpage

\section*{Oppgave 1}
  \subsection*{a)}
  Brewster vinkelen er vinkelen hvor en bølge treffer et nytt medium innfallsvinkelen står 90 grader på den transmitterte bølgen. 

  \subsection*{b)}
  Poyntingvektoren viser størrelse og retning på energitettheten til en elektromagnetiske bølge. Enheten er $W / m^2$

\subsection*{c)}
Planbølgen er gitt ved $f(x,t) = A \cos (kx - ωt + ϕ)$

\subsection*{d)}
\paragraph*{Konstant}
\begin{itemize}
    \item Frekvens
\end{itemize}

\paragraph*{Endres}
\begin{itemize}
    \item Bølgehastighet
    \item Bølgelengde 
    \item Utslag 
\end{itemize}

\subsection*{e)}
Vi bruker formelen i likning 10.15/16 i POW. Ettersom $θ_i = 90^{∘}$ dropper vi leddet med sinus. 
Da blir $R_s = \left(n_1 -n_2 / n_1 + n_2\right)^2$

\section*{Oppgave 2}
\subsection*{a)}
\colorbox{red}{Usikker}
Vi kan bestemme brennvidden ved å bruke linseformelen. 
\[
\frac{1}{f} = \frac{1}{s} + \frac{1}{s'} = (n-1) \left( \frac{1}{R_1} - \frac{1}{R_2} \right)
\]
for å finne brennvidden $f$. Det enkleste er å ha en objektavstand $s$ veldig langt unna slik at $1 / s ≈ 0$ som gjør at $f ≈ s'$, hvor $s'$ er bildeavstanden. Da er det bare å se hvor langt unna linsen man må se for å få et skarpt bilde. 

\subsection*{b)}
Vi bruker linsemakerformelen for å finne krummningsradien $R_c$
\[
\frac{1}{f} = (n-1) \left( \frac{1}{R_1} - \frac{1}{R_2} \right)
\]
Ettersom linsen er symmetrisk vet vi at $R_1 = - R_2$. Lyset beveger seg med en hastighet $c = 2c_0 / 3$ som gir $n = c_0 / c = 3 / 2$. 
\[
R_c = 2(3 / 2 - 1)f = 25 \text{mm}
\]

\subsection*{c)}
Dispersjon er når bølgehastigheten til en bølge ikke lenger er konstant, men avhengig av bølgelengden. Dette er ikke til å unngå i en lupe og gjør at forskjellige bølgelengder vil ha litt forskjellig brennvidde

\subsection*{d)}
Ved store mengder dispersjon vil du få forskjellig brennvidde avhengig av bølgelengden. Da kan bilde bli uskarpt. 

\subsection*{e)}
Vi bruker Rayleighs oppløsningskriterium. 
\[
Ψ = \frac{1.22 λ}{D}
\]
hvor $Ψ$ er vinkelen mellom de to objektene, $λ$ er bølgelengden og $D$ er diameteren til linsen.
\[
D = \frac{1.22 ⋅ 500 \text{ nm}}{9.7 ⋅ 10^{-12}} = 69 \text{ km}  
\]

 
\section*{Oppgave 3}
\subsection*{a)}
Kreftene som virker på steinen er luftmotstand, og tyngdekraften. Når steinen flyr forbi jordas sentrum vil tyngdekraftvektoren skifte retning, og fortsette å peke mot jordas sentrum. 

\subsection*{b)}
Vi ser at systemet kan sees på som en harmonisk oscillator, hvor $r = -ω^2 \ddot{r}$ og prøver derfor å sette $r = A\cos (ωt + ϕ)$. Ved å initialbetingelsene $\dot{r}(0) = 0 ⇒ ϕ = 0$ og $r(0) = R_e ⇒ A = R_e$ får vi følgende utrykk for $r$
\[
r = R_e \cos \left(\sqrt{\frac{3πGρ_0}{4}}t\right)
\]




\section*{Oppgave 4}
\subsection*{a)}
x-aksen vil vise tiden, og y-aksen viser intensiteten på lyden. 

\subsection*{b)}
Den høyeste frekvensen vi kan få er nyquist frekvensen på 22050 Hz. Fourier transformasjonen vil ha en skala fra 0, til 22050 Hz på x-aksen. Den vil ha en topp på 1300 Hz basert på at vi ser 13 topper, per 10 millisekund. 

\subsection*{c)}
Vi lager et spektogram ved å dele opp lydsignalet i mindre deler. Videre analyseres filen, og vi får en array med frekvenser og intensiteter. Dette plotter vi i et heatmap, for hver tids seksjon. Dette gjentas for hele lydfilen. 

\subsection*{d)}
x-aksen skal vise tid, og y-aksen skal vise frekvens. Colorbaren skal vise intensitet. 

\subsection*{e)}
Forskjellen mellom de to spektrogrammeneer at den første har et mye større tidsvindu enn den andre. Dette førere til høyre oppløsning av frekvens. Det motsatte stemmer for det andre bildet. 


  
  
  
  

\end{document}