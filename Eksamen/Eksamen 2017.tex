\documentclass{article}
\usepackage{amsmath}
\usepackage[mathletters]{ucs}
\usepackage[utf8x]{inputenc}
\usepackage[margin=1.5in]{geometry}
\usepackage{enumerate}
\newtheorem{theorem}{Theorem}
\usepackage[dvipsnames]{xcolor}
\usepackage{pgfplots}
\setlength{\parindent}{0cm}
\usepackage{graphics}
\usepackage{graphicx} % Required for including images
\usepackage{subcaption}
\usepackage{bigintcalc}
\usepackage{pythonhighlight} %for pythonkode \begin{python}   \end{python}
\usepackage{appendix}
\usepackage{arydshln}
\usepackage{physics}
\usepackage{tikz-cd}
\usepackage{booktabs} 
\usepackage{adjustbox}
\usepackage{mdframed}
\usepackage{relsize}
\usepackage{physics}
\usepackage[thinc]{esdiff}
\usepackage{fixltx2e}
\usepackage{esint}  %for lukket-linje-integral
\usepackage{xfrac} %for sfrac
\usepackage{hyperref} %for linker, må ha med hypersetup
\usepackage[noabbrev, nameinlink]{cleveref} % to be loaded after hyperref
\usepackage{amssymb} %\mathbb{R} for reelle tall, \mathcal{B} for "matte"-font
\usepackage{listings} %for kode/lstlisting
\usepackage{verbatim}
\usepackage{graphicx,wrapfig,lipsum,caption} %for wrapping av bilder
\usepackage{mathtools} %for \abs{x}
\usepackage[norsk]{babel}
\definecolor{codegreen}{rgb}{0,0.6,0}
\definecolor{codegray}{rgb}{0.5,0.5,0.5}
\definecolor{codepurple}{rgb}{0.58,0,0.82}
\definecolor{backcolour}{rgb}{0.95,0.95,0.92}
\lstdefinestyle{mystyle}{
    backgroundcolor=\color{backcolour},   
    commentstyle=\color{codegreen},
    keywordstyle=\color{magenta},
    numberstyle=\tiny\color{codegray},
    stringstyle=\color{codepurple},
    basicstyle=\ttfamily\footnotesize,
    breakatwhitespace=false,         
    breaklines=true,                 
    captionpos=b,                    
    keepspaces=true,                 
    numbers=left,                    
    numbersep=5pt,                  
    showspaces=false,                
    showstringspaces=false,
    showtabs=false,                  
    tabsize=2
}

\lstset{style=mystyle}
\author{Oskar Idland}
\title{Eksamen 2017}
\date{}
\begin{document}
\maketitle
\newpage
\section*{Oppgave 1}
  \subsection*{a)}
    Vi løser følgende likning.
    \[
    23 \text{dB} = 10 \log_{10} \left(I_2 / I_1\right) ⇒ 10^{23/10} = I_2 / I_1 ⇒ I_2 = 10^{23/10} I_1
    \]
    \[
    I_2 ≈ 200 I_1
    \]
  
  \subsection*{b)}
    Fjæringen til en bil er ikke underkritisk ettersom det hadde ført til at bilen fortsetter å bevege seg opp og ned etter å ha passert ulendt terreng. Fjæringen er kritisk ettersom man vil dempe svingingen så fort fort som mulig. 
    
  \subsection*{c)}
    Retningen til Poyntingvektoren er orientert ortogonalt på den elektriske feltvektoren. Det er fordi den beskriver hvor energien reiser hen. Bølgen selv oscillerer opp og ned ortogonalt på dette. 
    
  \subsection*{d)}
    Nei, totalrefleksjon kan bare oppnås ved å gå fra et medium med høyere brytningsindeks, til et medium med lavere brytningsindeks. Luft har laver brytningsindeks enn vann. 
  \subsection*{e)}
    Q faktor beskriver energitapet i et system. Dermed vil gitaren hvor lyden varer lengst ha høyest Q faktor. 
    
  \subsection*{f)}
    Vi beskriver systemet som en harmonisk oscillator. Det gir oss følgende sammenheng mellom posisjon og frekvens. 
    \[
    r = -ω^2 \ddot{r}
    \]
    Vi finner først vinkelfrekvensen $ω = 2πf = 2π ⋅ 2.55$ Hz. $ω = 16$ Hz. Videre løser vi for $ϕ$
    \[
    r = A \cos \left(ωt + ϕ\right) 
    \]
    Vi vet at når $t=0$ vil hastigheten også være 0. 
    \[
    \dot{r} = 0 → Aω \sin ϕ = 0 
    \]
    Ettersom både $A$ og $ω$ ikke er 0, må $ϕ = 0$. $A$ er amplituden og er oppgitt til å være 1.27 cm. Da har vi et fullstendig utrykk for posisjon som vi bruker til å finne maksimal hastighet og akselerasjon. 
    \[
    \dot{r}_{\text{max}} = \dot{r} (ωt = π/ 2) = Aω = 1.27 ⋅ 2.55 = 3.24 \text{ cm/s}
    \]
    Vi gjør det samme for akselerasjonen.
    \[
    \ddot{r}_{\text{max}} = \ddot{r} (ωt = π) = Aω^2 = 1.27 ⋅ 2.55^2 = 8.26 \text{ cm/s}^2
    \]
    
  \subsection*{g)}
    Fokallengden er gitt ved både diffraksjonsindeksen til mediet, og materialet til linsen. Da vil den endres når du setter linsen i vann. 
    
  \subsection*{h)}
    Avstanden mellom intensitetmaksima er invers proporsjonal med bølgelengden og du vil dermed ser større avstand ved rødt lys, enn ved blått. 
    
  \subsection*{i)}
    Fjernkontrollen bruker lys med frekvens på 40 MHz. Denne frekvensen reiser helt fint gjennom luft, men har mye kortere skinndybde i vann. Derfor er det lite gunstig å bruke trådløs fjernkontroll under vann. 
  
  \subsection*{j)}
    Frekvensen holder seg konstant. Alt annet endres ved endring av medium. Det er fordi bølgelengden er avhengig av bølgehastigheten. Utslaget vil naturligvis også endres. Stivere materialer vil kreve mer energi for å oppnå samme utslag.
    
  \subsection*{k)}
    Lyset fra sola har en gjennomsnittlig bølgelengde på $λ = 500$ nm. For å vise minst fem interferens striper bruker vi formelen for interferensmaksima.
    

\section*{Oppgave 2}
  \subsection*{a)}
    Vi antar herr Hook er et pendel med masse på $80$ kg. Vi vet maksimal amplitude $A=9$ cm. Vi vet at Hooks lover sier at $F = -kx$. Vi vet at $F_{\text{Hook}} = mg$ og $x_{\text{max}} = A$ og at $ΣF = 0$ ved $x_{\text{max}}$. Dermed kan vi finne fjærkonstanten $k$. 
    \[
    k = mg / A = 9.81 ⋅ 80 / 0.09 = 8720 \text{ N/m}
    \]
    Videre vet vi at svingefrekvensen $f = ω / 2π = \sqrt{k / m} / 2π = 0.74$ Hz. Dette er en underkritisk dempet svinging.   
  
  \subsection*{b)}
    Ved bølgelengde og frekvens kan vi finne bølgehastigheten.
    \[
    ν = fλ = 0.74 ⋅ 1.5 = 1.11 \text{ m/s}
    \]
    
  \subsection*{c)}
    Ved små vinkler kan vi tilnærme $\tanh θ = θ$. Når bølgen nærmer seg land vil høyden $h$ bli så liten at vi kan bruke denne approksimasjonen. Da ser vi raskt at når $h$ minker, vil også $ω$ minke. Dermed vil frekvensen $f$ minke som gjør at bølgehastigheten minke. 
    
  \subsection*{d)}
    Frekvens vil minke når $h$ minker når bølgen nærmer seg bredden. 
    
  \subsection*{e)}
    Ettersom bølgehastigheten minker, når bølgen nærmer seg bredden vil vi få en situasjon som minner litt om når lys går fra et medie med lav brytningsindeks til et medie med høy brytningsindeks. Da minker innfallsvinkelen. Det samme skjer med bølgene som nærmer seg bredden. En annen god metafor en når en gressklipper går fra asfalt, til gress med en innfallsvinkel. Gresset gir mer motstand til hjulene som treffer gresset først. De roterer dermed tregere og hjulene som treffer gresset sist vil rotere raskere i forhold. Det fører til rotasjon av gressklipperen så vinkelen mot infallsloddet blir redusert. 
    
    
\section*{Oppgave 3}
  \subsection*{a)}
    Vi ønsker at lyset fra objektet langt unna skal treffe skarpt på filmen. Da må filmen ha en avstand lik brennvidden $f = 400$ mm. 
    
    For et motiv bare 4.2 m unna må vi bruke linsemakerformelen. 
    \[
    \frac{1}{f} = \frac{1}{s'} + \frac{1}{s}
    \]
    der $f$ er brennvidden, $s'$ er bildeavstanden og $s$ er objektivavstanden. 
    \[
    \frac{1}{0.4 \text{ m}} = \frac{1}{s'} + \frac{1}{4.2 \text{ m}} ⇒ s' = 44 \text{ cm}
    \]
    Avstanden mellom filmen og linsen må være $44$ cm. 
    
    Vi sjekker avstandene med en objektiv avstand på 60 cm. 
    
    \[
    \frac{1}{s'} = \frac{1}{0.4 \text{ m}} - \frac{1}{0.6 \text{ m}} ⇒ s' = 1.2 \text{ m}
    \]
    Filmen må være veldig langt unna for å ta et bilde av noe så nære. Dette vil være vanskelig å få til. Hvertfall med et kamera bygd for en brennvidde på 400 mm.
    
  \subsection*{b)}
  
  
  \subsection*{f)}
    Forskjellige farger har forskjelige bølgelengder og dermed hastighet i mediet. Dermed brytes de litt forskjellig i linsen og dette gir et litt mer uklart motiv. 
    
  \subsection*{g)}
    Geometrisk optikk antar at lysstrålene ligger nære den optiske aksen. Det er en forutsetinging for å bruke linsemakerformelen. Da blir beregningen mest feil, i randen av linsen og vi ser ting litt uklart. 
    
    

\section*{Oppgave 4}
  \subsection*{a)}
    
    
    




\end{document}