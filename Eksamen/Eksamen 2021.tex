\documentclass{article}
\usepackage{amsmath}
\usepackage[mathletters]{ucs}
\usepackage[utf8x]{inputenc}
\usepackage[margin=1.5in]{geometry}
\usepackage{enumerate}
\newtheorem{theorem}{Theorem}
\usepackage[dvipsnames]{xcolor}
\usepackage{pgfplots}
\setlength{\parindent}{0cm}
\usepackage{graphics}
\usepackage{graphicx} % Required for including images
\usepackage{subcaption}
\usepackage{bigintcalc}
\usepackage{pythonhighlight} %for pythonkode \begin{python}   \end{python}
\usepackage{appendix}
\usepackage{arydshln}
\usepackage{physics}
\usepackage{tikz-cd}
\usepackage{booktabs} 
\usepackage{adjustbox}
\usepackage{mdframed}
\usepackage{relsize}
\usepackage{physics}
\usepackage[thinc]{esdiff}
\usepackage{fixltx2e}
\usepackage{esint}  %for lukket-linje-integral
\usepackage{xfrac} %for sfrac
\usepackage{hyperref} %for linker, må ha med hypersetup
\usepackage[noabbrev, nameinlink]{cleveref} % to be loaded after hyperref
\usepackage{amssymb} %\mathbb{R} for reelle tall, \mathcal{B} for "matte"-font
\usepackage{listings} %for kode/lstlisting
\usepackage{verbatim}
\usepackage{graphicx,wrapfig,lipsum,caption} %for wrapping av bilder
\usepackage{mathtools} %for \abs{x}
\usepackage[norsk]{babel}
\definecolor{codegreen}{rgb}{0,0.6,0}
\definecolor{codegray}{rgb}{0.5,0.5,0.5}
\definecolor{codepurple}{rgb}{0.58,0,0.82}
\definecolor{backcolour}{rgb}{0.95,0.95,0.92}
\lstdefinestyle{mystyle}{
    backgroundcolor=\color{backcolour},   
    commentstyle=\color{codegreen},
    keywordstyle=\color{magenta},
    numberstyle=\tiny\color{codegray},
    stringstyle=\color{codepurple},
    basicstyle=\ttfamily\footnotesize,
    breakatwhitespace=false,         
    breaklines=true,                 
    captionpos=b,                    
    keepspaces=true,                 
    numbers=left,                    
    numbersep=5pt,                  
    showspaces=false,                
    showstringspaces=false,
    showtabs=false,                  
    tabsize=2
}

\lstset{style=mystyle}
\author{Oskar Idland}
\title{Eksamen 2021}
\date{}
\begin{document}
\maketitle
\newpage

\section*{Flervalg}
  \begin{enumerate}
    \item Bølgehastighet 
      \[
      ν = λf
      \]
      \[
      f = \frac{1}{T} = \frac{1}{2} \quad , \quad λ = \frac{2π}{k} = π / 5
      \]
      \[
      ν = \frac{π}{5} \cdot \frac{1}{2} = \frac{π}{10} \text{ m/s}
      \] 


      \item Numeriske metoder
        \paragraph*{Riktige påstand: Nederst}
        I fjerde orden s Runke-Kutta-metode regner man ut flere estimater av stigningen og bruker et vektet gjennomsnitt av dem for å regne ut neste punkt. 
    

    \item Demping 
      \paragraph*{Riktig påstand: Nederst}
      Grønn (stiplet med prikk) er underkritisk dempet. Orange (stiplet) er kritisk dempet og blå (hel) er overkritisk dempet. 
    

    \item Vannbølger og EM-bølger 
      \paragraph*{Riktig påstand: Nest øverst}
      Overflate bølger på grunt vann er ikke dispersive. 
    

    \item Svingeligning 
      \[
      \ddot{x} = -kx - b \dot{x}
      \]
      beskriver en svingende bevegelse


    \item Strålegang % TODO: Dobbeltsjekk
      Figur b viser riktig diagram. Det er fordi strålen må alltid gå gjennom brennpunktet. 
      
    
    \item Samplingsfrekvens 
      Samplingsfrekvensen må minst være dobbelt av ønsket frekvens så vi trenger en samplingsfrekvens på minst 200 Hz.
      
    
    \item Dupp på overflaten til vann 
      Av å måle høyden til duppen kan vi vite perioden, som vi kan bruke til å finne \textbf{frekvensen}, 
    
    \item Løsning av bølgelikningen % TODO: Dobbeltsjekk
      \paragraph*{Feil likning: Nest Nederst}
      Begge cosinus ledd i bølgelikningen må ha samme faseforskyvning. 
      
    
    \item Frekvensspektrum % TODO: Dobbeltsjekk
      Vi finner grunnfrekvensen ved å se på avstanden mellom toppene i frekvensspekteret. Da er grunnfrekvensen 550-600 Hz. Intensiteten for hele trompeten ligger i intervallet 79-82 dB. Jeg antar dette fordi grunnfrekvensen bare er en del av den totale intensiteten. Dermed bør ikke totalen være lavere enn grunnfrekvensen.
      
  \end{enumerate}

\end{document}