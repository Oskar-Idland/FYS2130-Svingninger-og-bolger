\documentclass{article}
\usepackage{amsmath}
\usepackage[mathletters]{ucs}
\usepackage[utf8x]{inputenc}
\usepackage[margin=1.5in]{geometry}
\usepackage{enumerate}
\newtheorem{theorem}{Theorem}
\usepackage[dvipsnames]{xcolor}
\usepackage{pgfplots}
\pgfplotsset{compat=1.18}
\setlength{\parindent}{0cm}
\usepackage{graphics}
\usepackage{graphicx} % Required for including images
\usepackage{subcaption}
\usepackage{bigintcalc}
\usepackage{pythonhighlight} %for pythonkode \begin{python}   \end{python}
\usepackage{appendix}
\usepackage{arydshln}
\usepackage{physics}
\usepackage{tikz-cd}
\usepackage{booktabs} 
\usepackage{adjustbox}
\usepackage{mdframed}
\usepackage{relsize}
\usepackage{physics}
\usepackage[thinc]{esdiff}
\usepackage{fixltx2e}
\usepackage{esint}  %for lukket-linje-integral
\usepackage{xfrac} %for sfrac
\usepackage{hyperref} %for linker, må ha med hypersetup
\usepackage[noabbrev, nameinlink]{cleveref} % to be loaded after hyperref
\usepackage{amssymb} %\mathbb{R} for reelle tall, \mathcal{B} for "matte"-font
\usepackage{listings} %for kode/lstlisting
\usepackage{verbatim}
\usepackage{graphicx,wrapfig,lipsum,caption} %for wrapping av bilder
\usepackage{mathtools} %for \abs{x}
\usepackage[norsk]{babel}
\definecolor{codegreen}{rgb}{0,0.6,0}
\definecolor{codegray}{rgb}{0.5,0.5,0.5}
\definecolor{codepurple}{rgb}{0.58,0,0.82}
\definecolor{backcolour}{rgb}{0.95,0.95,0.92}
\lstdefinestyle{mystyle}{
    backgroundcolor=\color{backcolour},   
    commentstyle=\color{codegreen},
    keywordstyle=\color{magenta},
    numberstyle=\tiny\color{codegray},
    stringstyle=\color{codepurple},
    basicstyle=\ttfamily\footnotesize,
    breakatwhitespace=false,         
    breaklines=true,                 
    captionpos=b,                    
    keepspaces=true,                 
    numbers=left,                    
    numbersep=5pt,                  
    showspaces=false,                
    showstringspaces=false,
    showtabs=false,                  
    tabsize=2
}

\lstset{style=mystyle}
\author{Oskar Idland}
\title{Eksamen 2018}
\date{}
\begin{document}
\maketitle
\newpage

\section*{Oppgave 1}
  \subsection*{a)}
    Ettersom brytningsindeksen $n$ er gitt som $n = c_0 / v$, hvor $c_0$ er lysets hastighet i vakuum, og $v$ er hastigheten til lys i det aktuelle mediet, vil dette forholdet og dermed brytningsindeksen alltid være større enn 1. Det er fordi lyset alltid beveger seg raskere i vakuum, enn i et mediet. Det er fordi lys er en elektromagnetisk bølge som interagerer med de små elektromagnetiske feltene til partiklene i mediet. Denne forstyrrelsen skaper destruktiv interferens, ved samme frekvens, men med en liten forsinkelse. Dette gjør at gruppehastigheten til lyset blir tregere. 
    
  \subsection*{b)}
    Desibel skalaen er logaritmisk, så en økning på 4 dB øker intensiteten med en faktor på $10^{4}$. Det ser vi ved å løse følgende likning:
    \[
    4dB = 10 \log_{10} \left(\frac{I}{I_0}\right) \implies 10^{4/10} = \frac{I}{I_0} \implies I = 10^{4/10} I_0
    \]
  
  \subsection*{c)}
    \[
    ∇ × \vec{H} = \vec{j}_f + \frac{∂ \vec{D}}{∂ t}
    \]
    
  \subsection*{d)}
    Forskjellen i en Fourier transformasjon og en Wavelet transformasjon er at ved store endringer i frekvens, vil det lønne seg å dele opp signalet i kortere tidsperioder og utføre en FFT på bare denn tiden. En Fourier transformasjon tar inn hele signalet på en gang, mens et Wavelet transformasjon deler opp signalet i mindre vinduer som kan overlappe eller ikke. Her må en velge mellom å ha god tids oppløsning, eller frekvens oppløsning ettersom de er omvendt proporsjonale. en Fourier transformasjon plottes gjerne som en vanlig graf, men frekvens på $x$-aksen og amplitude på $y$-aksen.  En Wavelet transformasjon plottes som et spektrogram, hvor frekvens er på $y$-aksen, tid på $x$-aksen og farge intensitet er amplituden.
    
  \subsection*{e)}
    Lys brytes fra innfallsloddet når det krysser fra glass til luft. Det skjer alltid når lyset går fra en høyere brytningsindeks $n_1$ til en lavere indeks $n_2$. 
  
  \subsection*{f)}
    Ved laver bølgelengde, vil en få et tettere interferens mønster og dermed vil blått lys ha tettere samling av lys-maxima enn rødt lys. 
    
  \subsection*{g)}
    Ved å bare se på absoluttverdien av en Fouriertransformasjon, vil man miste faseinformasjonen for de forskjellige komponentene
  
  \subsection*{h)}
    En flat glassrute danner et bilde som er en-til-en med objektet. Det betyr at lyset passerer rett gjennom glasset og lager et bilde på størrelse med objektet. 
    
  \subsection*{i)}
    Hvis vi antar at lyset som kommer inn i kikkerten har en gjennomsnittlig bølgelengde på $λ = 500$ kan vi bruke følgende formel for å finne angulær distansen $Ψ$
    \[
    Ψ = \frac{1.22 λ}{D} = \frac{1.22 ⋅ 500 ⋅ 10^{-9} \text{m}}{0.1 \text{m}} = 6.1 ⋅ 10^{-6} \text{rad} 
    \]
    Nå som vi kjenner angulær distansen kan vi finne hvor stor lengde dette er ved 2 km avstand. 
    \[
    L = 2 ⋅ 10^3 \sin (Ψ) = 1.22 \text{ cm}
    \]
    Det minste objektet vi kan se har en høyde på $1.22$ cm. 
  
  \subsection*{j)}
    Poyntingsvektoren beskriver energitettheten til et elektromagnetisk bølge. Den har både størrelse og retning som viser hvor energien reiser. Den har enhetene $W / m^2$. 
    
  \subsection*{k)}
    Hvis vi ser på formelen for brennvidde $f$
    \[
    \frac{1}{f} = (n_{\text{linse}} -n_{\text{medium}}) \left(\frac{1}{R_1} - \frac{1}{R_2}\right)
    \]
    er brennvidden $f$ avhengig av begge mediene sin brytningsindeks. Dermed er brennvidden til en linse annerledes i luft, enn i vann.  
    

\section*{Oppgave 2}
  \subsection*{a)}
    Vi ser at dette kan beskrives som en harmonisk svinging. Vi starter med å finne $F_{\perp}$. 
    \[
    F_{\perp} = F_t \sin θ = -mg \sin θ
    \]
    

\section*{Oppgave 3}
  \subsection*{a)}


\section*{Oppgave 4}
  \subsection*{a)}
    Et dispersivt medie gjør at lyset reiser med forskjellig hastighet, avhengig av bølgelengde. Det gjør at diffraksjonsindexen blir litt forskjellig, som påvirker den nye vinkelen lyset har med innskuddsloddet. 
    
  \subsection*{b)}
    Dispersjon er hvordan et prisme deler opp en stråle av hvitt lys i flere farger. Hvitt lys inneholder alle farger, når det treffer prisme, vil alle de forskjellige bølgende i det hvite lyset oppleve litt forskjellig diffraksjonsindex, og dermed få litt forskjellig vinkel. Når de kommer ut av prismet igjen, vil de ha litt forskjellig posisjon, og vi kan se hver farge for seg. 
    
  \subsection*{c)}
    Ettersom vinkelhastighet $ω$ er gitt ved $ω = 2πf$, som betyr det er en funksjon av frekvensen $f$. Når en bølge reiser i et nytt medie, endrer hastigheten seg, og dermed må frekvensen også endres, ettersom bølgelengden $λ$ er konstant.
    
  \subsection*{d)}
    \[
    v_g = \frac{∂ ω}{∂ k} = c\sqrt{1 + β^2 k^2} + ck \frac{β^2k}{\sqrt{1 + β^2 k^2}} = c\frac{1 + 2β^2k^2}{\sqrt{1 + β^2k^2}}
    \]
    \[
    v_p = \frac{ω}{k} = c\sqrt{1 + β^2k^2}
    \]
    
  \subsection*{e)}
    Gruppehastigheten øker når bølgetallet $k$ øker. Bølgetallet øker når bølgelengden $λ$ minker. Det betyr at gruppehastigheten øker når bølgelengden minker.
    
  \subsection*{f)}
    En bølge som beveger seg langs en stiv streng vil bevege seg med gruppehastigheten og reflekteres tilbake når den når et eventuelt endepunkt. 
    
  \subsection*{g)}
    En kan ikke bruke monokromatisk lys for å se om et medium er dispersivt. Alt lyset vil reise med samme hastighet. 

    
    
       

\end{document}